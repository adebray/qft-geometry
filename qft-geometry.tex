\documentclass[12pt,letterpaper,reqno]{article}

% \usepackage{mathtools}
\usepackage{epsfig}
\usepackage{amsmath}
\usepackage{amssymb}
\usepackage{amsthm}
\usepackage{indentfirst}
\usepackage{xspace}
\usepackage{multirow}
\usepackage{hyperref}
\usepackage{xcolor}
\usepackage{verbatim}
\usepackage[letterpaper,margin=1in,headheight=15pt]{geometry}
\usepackage{mathpazo}
\usepackage{tikz-cd}
\usepackage{booktabs}
\usepackage{framed}
\usepackage{float}
\usepackage{thmtools}
\usepackage{dashrule}
\usepackage[missing=]{gitinfo2}
\usepackage{fancyhdr}

\definecolor{darkblue}{rgb}{0.1,0.1,0.7}
\definecolor{darkred}{rgb}{0.5,0.1,0.1}
\definecolor{darkgreen}{rgb}{0.0,0.42,0.06}
\hypersetup{colorlinks=true,urlcolor=darkred,linkcolor=darkblue,citecolor=darkred}
\definecolor{shadecolor}{rgb}{0.85,0.85,0.85}

% Bibliography formatting
\usepackage[bibstyle=authoryear-comp,labeldate=false,defernumbers=true,maxnames=20,uniquename=init,dashed=false,backend=biber,sorting=none]{biblatex}

\DeclareNameAlias{sortname}{first-last}

\DeclareFieldFormat{url}{\url{#1}}
\DeclareFieldFormat[article]{pages}{#1}
\DeclareFieldFormat[inproceedings]{pages}{\lowercase{pp.}#1}
\DeclareFieldFormat[incollection]{pages}{\lowercase{pp.}#1}
\DeclareFieldFormat[article]{volume}{\textbf{#1}}
\DeclareFieldFormat[article]{number}{(#1)}
\DeclareFieldFormat[article]{title}{\MakeCapital{#1}}
\DeclareFieldFormat[inproceedings]{title}{#1}
\DeclareFieldFormat{shorthandwidth}{#1}

% Don't use "In:" in bibliography. Omit urls from journal articles.
\DeclareBibliographyDriver{article}{%
  \usebibmacro{bibindex}%
  \usebibmacro{begentry}%
  \usebibmacro{author/editor}%
  \setunit{\labelnamepunct}\newblock
  \MakeSentenceCase{\usebibmacro{title}}%
  \newunit
  \printlist{language}%
  \newunit\newblock
  \usebibmacro{byauthor}%
  \newunit\newblock
  \usebibmacro{byeditor+others}%
  \newunit\newblock
  \printfield{version}%
  \newunit\newblock
%  \usebibmacro{in:}%
  \usebibmacro{journal+issuetitle}%
  \newunit\newblock
  \printfield{note}%
  \setunit{\bibpagespunct}%
  \printfield{pages}
  \newunit\newblock
  \usebibmacro{eprint}
  \newunit\newblock
  \printfield{addendum}%
  \newunit\newblock
  \usebibmacro{pageref}%
  \usebibmacro{finentry}}

% Remove dot between volume and number in journal articles.
\renewbibmacro*{journal+issuetitle}{%
  \usebibmacro{journal}%
  \setunit*{\addspace}%
  \iffieldundef{series}
    {}
    {\newunit
     \printfield{series}%
     \setunit{\addspace}}%
  \printfield{volume}%
%  \setunit*{\adddot}%
  \printfield{number}%
  \setunit{\addcomma\space}%
  \printfield{eid}%
  \setunit{\addspace}%
  \usebibmacro{issue+date}%
  \newunit\newblock
  \usebibmacro{issue}%
  \newunit}


% Bibliography categories
\def\makebibcategory#1#2{\DeclareBibliographyCategory{#1}\defbibheading{#1}{\section*{#2}}}
\makebibcategory{books}{Books}
\makebibcategory{papers}{Refereed research papers}
\makebibcategory{chapters}{Book chapters}
\makebibcategory{conferences}{Papers in conference proceedings}
\makebibcategory{techreports}{Unpublished working papers}
\makebibcategory{bookreviews}{Book reviews}
\makebibcategory{editorials}{Editorials}
\makebibcategory{phd}{PhD thesis}
\makebibcategory{subpapers}{Submitted papers}
\makebibcategory{curpapers}{Current projects}

\setlength{\bibitemsep}{2.65pt}
\setlength{\bibhang}{.8cm}
\renewcommand{\bibfont}{\small}

\renewcommand*{\bibitem}{\addtocounter{papers}{1}\item \mbox{}\hskip-0.85cm\hbox to 0.85cm{\hfill\arabic{papers}.~~}}
\defbibenvironment{bibliography}
{\list{}
  {\setlength{\leftmargin}{\bibhang}%
   \setlength{\itemsep}{\bibitemsep}%
   \setlength{\parsep}{\bibparsep}}}
{\endlist}
{\bibitem}

\newenvironment{publications}{\section{\LARGE Publications}\label{papersstart}\vspace*{0.2cm}\small
\titlespacing{\section}{0pt}{1.5ex}{1ex}\itemsep=0.00cm
}{\label{papersend}\addtocounter{sumpapers}{-1}\refstepcounter{sumpapers}\label{sumpapers}}

\def\printbib#1{\printbibliography[category=#1,heading=#1]\lastref{sumpapers}}

% Counters for keeping track of papers
\newcounter{papers}\setcounter{papers}{0}
\newcounter{sumpapers}\setcounter{sumpapers}{0}
\def\lastref#1{\addtocounter{#1}{\value{papers}}\setcounter{papers}{0}}

% theorem environments
\declaretheoremstyle[spaceabove=0.25cm,spacebelow=0.25cm,notefont=\normalfont\bfseries, notebraces={(}{)}]{theorem}
\declaretheoremstyle[spaceabove=0.25cm,spacebelow=0.25cm,bodyfont=\normalfont,notefont=\normalfont\bfseries, notebraces={(}{)}]{noital}
\declaretheoremstyle[spaceabove=0.25cm,spacebelow=0.25cm,bodyfont=\normalfont\color{darkgreen},notefont=\normalfont\bfseries, notebraces={(}{)}]{green}
\declaretheoremstyle[spaceabove=0.25cm,spacebelow=0.25cm,bodyfont=\normalfont,notefont=\normalfont\bfseries,qed=$\qedsymbol$,notebraces={(}{)}]{proofstyle}

\declaretheorem[name=Theorem,numberwithin=section,style=theorem]{thm}
\declaretheorem[name=Fact,numberwithin=section,style=theorem]{fact}
\declaretheorem[name=Proposition,sibling=thm,style=theorem]{prop}
\declaretheorem[name=Corollary,sibling=thm,style=theorem]{cor}
\declaretheorem[name=Lemma,sibling=thm,style=theorem]{lem}
\declaretheorem[name=Definition,sibling=thm,style=noital]{defn}
\declaretheorem[name=Example,sibling=thm,style=noital]{example}
\declaretheorem[name=Exercise,numberwithin=section,style=green]{exercise}
\declaretheorem[name=Proof,style=proofstyle,numbered=no]{pf}

\numberwithin{equation}{section}


% macros for convenience
\newcommand{\tops}{\texorpdfstring}

\newcommand{\nid}{\noindent}

\newcommand{\fa}{{\mathfrak a}}
\newcommand{\fp}{{\mathfrak p}}
\newcommand{\fk}{{\mathfrak k}}
\newcommand{\fg}{{\mathfrak g}}
\newcommand{\fh}{{\mathfrak h}}
\newcommand{\fn}{{\mathfrak n}}
\newcommand{\fq}{{\mathfrak q}}
\newcommand{\fm}{{\mathfrak m}}
\newcommand{\fr}{{\mathfrak r}}
\newcommand{\fu}{{\mathfrak u}}
\newcommand{\fG}{{\mathfrak G}}

\newcommand{\cC}{\ensuremath{\mathcal C}}
\newcommand{\cG}{\ensuremath{\mathcal G}}
\newcommand{\cB}{\ensuremath{\mathcal B}}
\newcommand{\cL}{\ensuremath{\mathcal L}}
\newcommand{\cS}{\ensuremath{\mathcal S}}
\newcommand{\cF}{\ensuremath{\mathcal F}}
\newcommand{\cK}{\ensuremath{\mathcal K}}
\newcommand{\cZ}{\ensuremath{\mathcal Z}}
\newcommand{\cM}{\ensuremath{\mathcal M}}
\newcommand{\cN}{\ensuremath{\mathcal N}}
\newcommand{\cO}{\ensuremath{\mathcal O}}
\newcommand{\cH}{\ensuremath{\mathcal H}}
\newcommand{\cX}{\ensuremath{\mathcal X}}
\newcommand{\cY}{\ensuremath{\mathcal Y}}
\newcommand{\cA}{\ensuremath{\mathcal A}}
\newcommand{\cI}{\ensuremath{\mathcal I}}

\newcommand{\R}{\ensuremath{\mathbb R}}
\newcommand{\C}{\ensuremath{\mathbb C}}
\newcommand{\PP}{\ensuremath{\mathbb P}}
\newcommand{\Z}{\ensuremath{\mathbb Z}}
\newcommand{\Q}{\ensuremath{\mathbb Q}}
\newcommand{\A}{\ensuremath{\mathbb A}}
\newcommand{\bbH}{\ensuremath{\mathbb H}}
\newcommand{\bbI}{\ensuremath{\mathbb I}}
\newcommand{\bS}{\ensuremath{\mathbb S}}

\newcommand{\half}{\ensuremath{\frac{1}{2}}}
\newcommand{\qtr}{\ensuremath{\frac{1}{4}}}
\newcommand{\bq}{{\mathbf q}}
\newcommand{\N}{{\mathcal N}}
\newcommand{\F}{{\mathcal F}}
\newcommand{\HH}{{\mathcal H}}
\newcommand{\LL}{{\mathcal L}}
\newcommand{\RR}{{\mathcal R}}
\newcommand{\V}{{\mathcal V}}
\newcommand{\dirac}{\!\!\not\!\partial}
\newcommand{\Dirac}{\!\!\not\!\!D}
\newcommand{\cE}{{\mathcal E}}
\newcommand{\vs}{\not\!v}
\newcommand{\kahler}{K\"ahler\xspace}
\newcommand{\kq}{/\!\!/}
\newcommand{\kql}[1]{/\!\!/\!\!_#1\,}
\newcommand{\hk}{hyperk\"ahler\xspace}
\newcommand{\Hk}{Hyperk\"ahler\xspace}
\newcommand{\hkq}{/\!\!/\!\!/\!\!/}
\newcommand{\hkql}[1]{/\!\!/\!\!/\!\!/\!\!_#1\,}
\newcommand{\del}{\ensuremath{\partial}}
\newcommand{\delbar}{\ensuremath{\overline{\partial}}}
\newcommand{\I}{{\mathrm i}}
\newcommand{\J}{{\mathrm j}}
\newcommand{\K}{{\mathrm k}}
\newcommand{\e}{{\mathrm e}}
\newcommand\bid{{\mathbf 1}}
\newcommand{\de}{\mathrm{d}}
\newcommand{\ab}{\mathrm{ab}}
\newcommand{\vol}{\mathrm{vol}}
\renewcommand{\sf}{\mathrm{sf}}
\newcommand{\inst}{\mathrm{inst}}
\newcommand{\eff}{\mathrm{eff}}
\newcommand{\dR}{\mathrm{dR}}
\newcommand{\closed}{\mathrm{closed}}
\newcommand{\exact}{\mathrm{exact}}

\newcommand{\abs}[1]{\lvert#1\rvert}
\newcommand{\norm}[1]{\lVert#1\rVert}
\newcommand{\IP}[1]{\langle#1\rangle}
\newcommand{\DIP}[1]{\langle\!\langle#1\rangle\!\rangle}
\newcommand{\dwrt}[1]{\frac{\partial}{\partial#1}}
\newcommand{\eps}{\epsilon}
\newcommand{\simarrow}{\xrightarrow\sim}

\newcommand{\mmaref}[1]{}

\newcommand{\ti}[1]{\textit{#1}}
\newcommand{\tb}[1]{\textbf{#1}}

\DeclareMathOperator{\ad}{ad}
\DeclareMathOperator{\im}{Im}
\DeclareMathOperator{\re}{Re}
\DeclareMathOperator{\Tr}{Tr}
\DeclareMathOperator{\End}{End}
\DeclareMathOperator{\Hom}{Hom}
\DeclareMathOperator{\Aut}{Aut}
\DeclareMathOperator{\Sym}{Sym}
\DeclareMathOperator{\Lie}{Lie}
\DeclareMathOperator{\diag}{diag}
\DeclareMathOperator{\Bun}{Bun}
\DeclareMathOperator{\Vect}{Vect}
\DeclareMathOperator{\Span}{Span}
\DeclareMathOperator{\grad}{grad}
\DeclareMathOperator{\rank}{rank}
\DeclareMathOperator{\ind}{ind}
\DeclareMathOperator{\coker}{coker}
\DeclareMathOperator{\Jac}{Jac}
\DeclareMathOperator{\Hol}{Hol}
\DeclareMathOperator{\gr}{gr}

\newcommand{\insfig}[2]{

\medskip
\noindent
\begin{minipage}{\linewidth}

\makebox[\linewidth]{\includegraphics[keepaspectratio=true,scale=#2]{figures/#1-crop.pdf}}

\end{minipage}
\medskip

}


% \newcommand{\insfig}[2]{\begin{figure}[htbp] \centering \includegraphics[scale=#2]{figures/#1-crop.pdf} \label{fig:#1} \end{figure}}
% syntax: \insfig{name}{0.5}{caption}

\newcommand{\fixme}[1]{{\color{orange}{[#1]}}}
\newcommand{\currentposition}{{\color{blue} \noindent\makebox[\linewidth]{\hdashrule{\paperwidth}{1pt}{3mm}}}}

% \mathtoolsset{showonlyrefs}

\bibliography{qft-geometry}

\begin{document}

\pagestyle{fancy}
\lhead{{\tiny \color{gray} \tt \gitAuthorIsoDate}}
\chead{\tiny \ti{Applications of QFT to Geometry, \tb{preliminary} and \tb{incomplete} draft}}
\rhead{{\tiny \color{gray} \tt \gitAbbrevHash}}
\renewcommand{\headrulewidth}{0.5pt}


\begin{center}
\tb{Applications of QFT to Geometry} \\
Andrew Neitzke \\
\tb{Preliminary} and \tb{incomplete} draft
\end{center}

{These are the notes for a Fall 2017
course at UT Austin, \ti{in progress}. 
They are extremely incomplete, unreliable, full
of mistakes and omissions.
The latest PDF can always be found
at
\begin{center}
\small \url{http://ma.utexas.edu/users/neitzke/teaching/392C-qft-geometry/qft-geometry.pdf}
\end{center}
Please send corrections/improvements to
\begin{center}
\small \tt\href{mailto:neitzke@math.utexas.edu}{neitzke@math.utexas.edu}
\end{center}
or as pull requests to the source repository hosted at
\begin{center}
\small \url{http://github.com/neitzke/qft-geometry}
\end{center}
}

% \tableofcontents
% \renewcommand{\listtheoremname}{Quick reference}
% \listoftheorems[onlynamed]

% \newpage

%\setcounter{page}{1}

\section{Introductory motivation}

\subsection{Linear equations} 
Suppose we are interested in studying smooth manifolds $X$.
One simple and powerful tool for this purpose: fix a Riemannian metric $g$ on $X$.
This allows us to define the \ti{form Laplacian}
\begin{equation}
  \Delta = [\de, \de^*]
\end{equation}
and consider the space of \ti{harmonic forms}
\begin{equation}
  \cH^k(X) = \{ \omega \in \Omega^k(X) \vert \Delta \omega = 0 \}.
\end{equation}
The equation 
\begin{equation} \label{eq:laplace}
\Delta \omega = 0  
\end{equation}
is a \ti{linear} equation over $\R$.
Thus $\cH^k(X)$ is a \ti{vector space}.
If $X$ is compact, then $\cH^k(X)$ is moreover finite-dimensional (a consequence of
the fact that $\Delta$ is an \ti{elliptic} operator). Thus we can define
positive integers by
\begin{equation}
  b_k(X) = \dim_\R \cH^k(X).
\end{equation}
There is a remarkable fact about these integers:
\begin{fact} The integers $b_k$ do not depend on the choice of Riemannian metric or smooth
structure on $X$; instead they are invariants of the underlying topological manifold 
(the \ti{Betti numbers}).
\end{fact}

This is a consequence of a stronger, ``categorified'' fact:
\begin{fact} There is a canonical isomorphism
\begin{equation}
  \cH^k(X) \simeq H^k(X,\R)
\end{equation}
(singular cohomology).
\end{fact}

If $X$ is an oriented $4n$-manifold then there is a small refinement of the 
middle Betti number $b_{2n}$:
we have the ``Hodge star'' operator
\begin{equation}
  \star: \Omega^{2n}(X) \to \Omega^{2n}(X)
\end{equation}
which has the crucial properties 
\begin{itemize}
  \item $\star^2 = 1$,
  \item $[\star, \Delta] = 0$.
\end{itemize}
The first property says we can decompose into eigenspaces for $\star$:
\begin{equation}
  \Omega^n(X) = \Omega^{n,+}(X) \oplus \Omega^{n,-}(X)
\end{equation}
and the second says that the harmonic forms also decompose:
\begin{equation}
  \cH_{2n}(X) = \cH_{2n}^+(X) \oplus \cH_{2n}^-(X), \qquad b_{2n}(X) = b_{2n}^+(X) + b_{2n}^-(X).
\end{equation}

\begin{exercise} Suppose instead that $X$ has 
dimension $4n+2$. Here the story is a bit different, because in dimension $k = 2n+1$
we have $\star^2 = -1$.
Show that in this case the Betti number $b_{2n+1}(X)$ is even.
\end{exercise}



\subsection{Nonlinear equations}

A new source of topological (or more precisely smooth) invariants was discovered
by Donaldson in the 1980s. For an excellent reference see \cite{MR1079726}.

The first key idea is to replace the linear equation
\eqref{eq:laplace} by a \ti{nonlinear} equation.
Fix a compact Lie group $G$ and let $E$ denote a principal $G$-bundle over $X$. Then
we consider \ti{connections} in $E$. Such a connection is locally written
\begin{equation}
  D = \de + A, \qquad A \in \Omega^1(\fg),
\end{equation}
and has a curvature $2$-form $F \in \Omega^2(\ad E)$, locally written
\begin{equation} \label{eq:curvature}
  F = \de A + A \wedge A \in \Omega^2(\fg),
\end{equation}
which again we can decompose under $\star$ as
\begin{equation}
  F = F^+ + F^-.
\end{equation}
Now the \ti{self-dual Yang-Mills equation} is
\begin{equation} \label{eq:sdym}
  F^+ = 0.
\end{equation}
For $G$ abelian (e.g. for $G = U(1)$) the equation \eqref{eq:sdym} is linear.
% It simply says
% \begin{equation}
%   \de A = \star \de A.
% \end{equation}
% \begin{exercise}
% ...
% \end{exercise}
For $G$ nonabelian (e.g. for $G = SU(2)$) this
equation is \ti{nonlinear}, because of the quadratic part in \eqref{eq:curvature}.

\begin{exercise} Write out \eqref{eq:sdym} in detail in components, in two cases:
\begin{itemize}
  \item For $G = U(1)$: in this case you should get a system of $3$ linear equations for $4$ functions $A_i$
  ($i \in \{1,2,3,4\}$).
  \item For $G = SU(2)$: in this case you should get a system of $9$ coupled nonlinear equations for
  $12$ functions $A^a_i$ ($i \in \{1,2,3,4\}$, $a \in \{1,2,3\}$).
\end{itemize}
\end{exercise}

% From now on we specialize to the interesting case $G = SU(2)$.
We consider
\begin{equation}
  \cM = \{ \text{solutions of \eqref{eq:sdym}} \} / \fG
\end{equation}
where $\fG$ is the infinite-dimensional group of \ti{gauge transformations} ie
sections of $\Aut(E)$, acting on connections by
\begin{equation}
  D \mapsto g D g^{-1}.
\end{equation}

\begin{exercise}
Verify that $\fG$ acts on the space of solutions of \eqref{eq:sdym}. Write out this action 
explicitly in components, when $G = U(1)$ or $G = SU(2)$.
\end{exercise}

\begin{exercise}
Show that when $G = U(1)$, $\cM \simeq \cH_{2}^{+}(X)$; in particular, in this case 
$\cM$ actually \ti{is} a vector space. \fixme{warning: needs Hodge theory}
\end{exercise}

When $G$ is nonabelian, the situation is much more difficult.
$\cM$ is not a vector space. Nevertheless it has some reasonable geometric
structure:
\begin{fact}
When $G = SU(2)$, $\cM$ has the natural structure of a ``manifold away from some singular points''
\fixme{...}
\end{fact}

Donaldson's idea was to extract interesting invariants of $X$ by studying the space $\cM$. The 
\ti{dimension} of $\cM$ turns out not to be so interesting: it is determined by 
$c_2(E)$, which has little to do with $X$. What is more interesting is to compute 
certain \ti{integrals} over $\cM$.

\printbibliography

\end{document}